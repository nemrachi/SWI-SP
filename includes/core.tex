\section{Používateľská špecifikácia} % 3-4 strany
\noindent Verzia FEIstyle 1.5 používa glossary\footnote{\url{https://www.ctan.org/pkg/glossaries?lang=en}} balík.
\acrfull{cdma} je dlhá skratka naopak \acrshort{gsm} je skratka v krátkej forme.

Vžite sa do role majiteľa softvérovej firmy, ktorého zákazník požiadal o vytvorenie systému a vytvorte
používateľskú špecifikáciu, ktorá bude slúžiť ako súčasť zmluvy.

\subsection{Stručný úvod do problematiky}

Tu treba popísať, čo sa v danej oblasti robí, aké sú tam pravidlá,
ciele, postupy, aká je business logika (doménová logika) atď. Heslovite je táto informácia
zhrnutá v zadaní, ktoré ste dostali, avšak treba ju rozvinúť a doplniť nespomenuté časti a
súvislosti. Použite vlastnú inteligenciu, tvorivosť, externé zdroje a diskusiu s inými ľuďmi, aby
ste zistili, ako funguje daná doménová oblasť.

\subsection{Používateľské požiadavky}
Definujte zákazníkove ciele a prepíšte ich na merateľné požiadavky.
Identifikujte a popíšte funkcionálne, nefunkcionálne a doménové požiadavky.

%%%%%%%%%%%%%%%%%%%%%%%%%%%%%%%%%%%%%%%%%%%%%%%%%%%%%%%%%%%%%%%%%%%%%%%%

\section{Systémová špecifikácia}
V diagramoch použite notáciu UML verzie 2.x

\subsection{Diagramy prípadov použitia.}
Nakreslite diagram(y) prípadov použitia pre daný softvérový
systém. Diagram (minimálne jeden, prípadne viacej ak sa to hodí), bude pomocou prípadov
použitia obsahovať celú hlavnú funkcionalitu systému. Každý prípad použitia by mal, v rámci
svojej realizácie, poskytovať svojmu hráčovi (alebo hráčom) niečo hodnotné, nejakú užitočnú
funkcionalitu, nejaký pozorovateľný výsledok alebo zmenu.

\subsection{Use-case tabuľky}
K trom najzložitejším prípadom použitia vytvorte use-case tabuľku, ktorá
bude obsahovať [2b]:
– názov prípadu použitia
– identifikátor - ako identifikátor môžete použiť svoje vlastné číslovanie, ktoré bude
spájať jednotlivé prípady použitia z diagramu prípadov použitia.
– opis prípadu použitia (stručný)
– hráčov
– vstupné podmienky
– inicializácia
– hlavnú postupnosť udalostí
– alternatívnu postupnosť udalostí
– výstupné podmienky
VZOR: tutoriál č.2 – \href{https://uim.fei.stuba.sk/wp-content/uploads/2022/09/CV2-USC_tabulka_TUTORIAL.pdf}{use-case tabuľka}

\subsection{Diagram tried}
Vytvorte jeden detailný dátový model pre celý váš systém, ktorý bude zahŕňať
všetky atribúty, vzťahy, násobnosti a aspoň niektoré metódy. Zobrazte ho ako jeden UML 2.x
diagram tried vo vašej výslednej dokumentácii. Ak je systém príliš komplexný, môžete rozčleniť
diagram na viacero menších diagramov, ktoré budú reprezentovať len príslušný podsystém.

\subsection{Diagramy aktivít a sekvenčné diagramy}
K vybraným netriviálnym prípadom použitia nakreslite
diagramy graficky popisujúce tieto prípady použitia. Nakreslite 2 sekvenčné diagramy a 2
diagramy aktivít.

\subsection{Stavový diagram}
Nakreslite stavový diagram pre vami vyvíjaný systém a v tabuľkách popíšte
jednotlivé stavy a prechody. Môžete vytvoriť aj viacero menších stavových diagramov namiesto
jedného veľkého.

%%%%%%%%%%%%%%%%%%%%%%%%%%%%%%%%%%%%%%%%%%%%%%%%%%%%%%%%%%%%%%%%%%%%%%%%

\section{Akceptačné testy}
Vytvorte testy, na základe ktorých sa rozhodne o tom, či vytvorený systém spĺňa alebo nespĺňa
požiadavky – teda či ho zákazník akceptuje alebo nie. Každý test by mal v tabuľke obsahovať minimálne
tieto časti:
• identifikátor
• prípad použitia, ku ktorému test prislúcha
• cieľ testu (čo overujeme – nanajvýš stručne)
• vstupné podmienky
• výstupné podmienky
• jednotlivé kroky testu
Kroky testu reprezentujú sekvenciu testovania a ku každému kroku prislúcha a je v teste popísaná určitá
akcia (podnet od aktéra) a určitá reakcia systému na tento podnet. Aby bol výsledný systém zákazníkom
akceptovaný, musí splniť všetky testy. Keďže v tomto zadaní systém neprogramujeme ale len
navrhujeme, jednotlivé očakávané reakcie je potrebné si vymyslieť.
Do dokumentácie doplňte aspoň 5 akceptačných testov
• štyri, ktoré súvisia s funkcionálnymi požiadavkami a
• jeden, ktorý overuje nefunkcionálne požiadavky.
PRÍKLAD: \href{https://uim.fei.stuba.sk/wp-content/uploads/2022/10/AkceptacneTestyPriklad.pdf}{AkceptacneTestyPriklad.pdf}

%%%%%%%%%%%%%%%%%%%%%%%%%%%%%%%%%%%%%%%%%%%%%%%%%%%%%%%%%%%%%%%%%%%%%%%%

\section{Projektové plánovanie}
Vytvorte plán tvorby (realizácie) vášho systému.
1. Rozdeľte prácu na aspoň 10 úloh a rozdeľte úlohy pre aspoň 4 ľudí tvoriacich váš tím. Počet si zvoľte
podľa náročnosti témy, ale minimálne musí mať váš tím aspoň 4 členov.
2. Odhadnite časovú náročnosť úloh, naplánujte postupnosť úloh do kalendára.
3. V dokumente v kapitole 4.1 zobrazte Ganttov graf aj s tabuľkou závislostí a postupnosti vykonávania
úloh, s míľnikmi a s WBS (work breakdown schedule).
4. V dokumente v kapitole 4.2. zobrazte sieťový graf pre postupnosti vykonávania úloh.
Na túto úlohu použite vami zvolený systém na projektový manažment (či už offline, lokálny program
alebo ľubovoľný/dostupný online produkt). Zoznam je napr. na:
\url{http://en.wikipedia.org/wiki/Comparison_of_project-management_software}. Úlohou je oboznámiť sa
so systémom na projektový manažment.
Odporúčame: Microsoft Project, Project Libre alebo google: alternatives to ms project

%%%%%%%%%%%%%%%%%%%%%%%%%%%%%%%%%%%%%%%%%%%%%%%%%%%%%%%%%%%%%%%%%%%%%%%%

% \begin{table}[!htbp]
% \caption{Moduly a ich funkcie pri anonymizácii}
% \label{modulyVlastnosti}
% \begin{center}
% \begin{tabular}{p{4cm}|c|c|c|c|c|c|c|c|c|c|c|c|c|c|c}
% & \multicolumn{14}{c}%
% 	 {\textbf{Funkcia}}\\ \hline
% &&&& & &\multicolumn{8}{c}%
% 	 {Modifikácia}\\ 
% \textbf{Modul} &\begin{sideways} zobrazenie hlavičky \end{sideways} &\begin{sideways} blokovanie skriptov \end{sideways} &\begin{sideways} zmena IP \end{sideways} & \begin{sideways} zmena lokalizácie \end{sideways} & \begin{sideways} zmazanie/blokovanie cookies \end{sideways} & \begin{sideways} blokovanie trackerov \end{sideways}  & \begin{sideways} popis \end{sideways} & \begin{sideways}používateľský agent\end{sideways} & \begin{sideways} kódové označenie prehliadača \end{sideways} & \begin{sideways} názov prehliadača \end{sideways} & \begin{sideways} verzia prehliadača \end{sideways} & \begin{sideways} platforma \end{sideways} & \begin{sideways} výrobca prehliadača \end{sideways} & \begin{sideways} označenie výrobcu prehliadača \end{sideways} \\ \hline
% User agent switcher & & & & & &  & X & X & X & X & X & X & X & X  \\ \hline
% Ghostery &  && & & X & X &  &  & & & & & & \\  \hline
% Better privacy && &  & & X &  &  &  & & & & & & \\  \hline
% Anonymox &  && X & X & X &  & X & X & & & & & & \\  \hline
% Modify headers & & &  &  & X &  &  & X &  &  &  & & &  \\  \hline
% Request policy & & &  &  & & X  &  &  &  &  &  & & &   \\  \hline
% Live HTTP headers & X& &  &  & &  &  &  &  &  &  & & &   \\  \hline
% User agent awitcher & & &  &  & &  & X & X &  &  &  & & &   \\  \hline
% Header hacker & & &  &  & &  & X & X & X & X & X & X & X & X    \\  \hline
% Mod header & & &  &  & &  & X & X & X & X & X & X & X & X    \\  \hline
% Script no & &X &  &  & &  &  &  &  &  &  &  &  &     \\  \hline
% No script & &X &  &  & &  &  &  &  &  &  &  &  &     \\  \hline
% Proxify it & & &X  & X & &  &  &  &  &  &  &  &  &     \\  \hline
% I'm not here & & &  & X & &  &  &  &  &  &  &  &  &     \\  \hline
% Get edition & &X &X &X &X&X &  &  &  &  &  &  &  &     \\  \hline
% Anonymous browsing toolbar & & & X & X & &  &  &  &  &  &  &  &  &     \\  \hline
% Easy hide your IP and surf & & & X & X& &  &  & X & X & X & X &  &  &     \\  \hline
% \end{tabular}
% \end{center}
% \end{table}

% \begin{figure}[!htbp]
%   \centering
%   \includegraphics[width=8cm]{img/vzhlad.png}
%   \caption{Predpokladaný vzhľad rozšírenia.}
%   \label{vzhladobr}
% \end{figure}	 

% \begin{algorithm}
% \scriptsize
% \begin{algorithmic}
%  \STATE <text>
%  \IF{<condition>} \STATE {<text>} \ELSE \STATE{<text>} \ENDIF
%  \IF{<condition>} \STATE {<text>} \ELSIF{<condition>} \STATE{<text>} \ENDIF
%  \FOR{<condition>} \STATE {<text>} \ENDFOR
%  \FOR{<condition> \TO <condition> } \STATE {<text>} \ENDFOR
%  \FORALL{<condition>} \STATE{<text>} \ENDFOR
%  \WHILE{<condition>} \STATE{<text>} \ENDWHILE
%  \REPEAT \STATE{<text>} \UNTIL{<condition>}
%  \LOOP \STATE{<text>} \ENDLOOP
%  \REQUIRE <text>
%  \ENSURE <text>
%  \RETURN <text>
%  \PRINT <text>
%  \COMMENT{<text>}
%  \AND, \OR, \XOR, \NOT, \TO, \TRUE, \FALSE
% \end{algorithmic}
% \caption{Ukážka príkazov pre algorithmic}  
% \label{alg:preview}  
% \end{algorithm}

% \begin{lstlisting}[
%   caption={Ukážka algoritmu},
%   label={lst:main-c},
%   language=c
% ]
% /* Hello World program */

% #include<stdio.h>

% struct cpu_info {
%     long unsigned utime, ntime, stime, itime;
%     long unsigned iowtime, irqtime, sirqtime;
% };

% main()
% {
%     printf("Hello World");
% }
% \end{lstlisting}